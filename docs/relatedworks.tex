\documentclass{article}

\title{Related Works}
\author{Author}

\begin{document}

\maketitle

\section{ORB SLAM}

\subsection{Introduction}

Bundle adjustment is known to provide accurate extimates of camera localization as well as a sparse geometrical reconstruction.

Contributions:\\
1. Use of the same features for all tasks;\\
2. Real time operation of large environment;\\
3. Real time loop closing based on optimization of a pose graph;\\
4. Real time relocalization with signifiant invariance to viewport and illumination;\\
5. A new automatic and robust initialization procedure based on model selection that permits to create an initial map of planar and non-planar scene;\\
6. A survival of fittest approach to map point and keyframe selection that is generous in spawning but very restrictive in culling.

\subsection{Related Works}

A. Place Recognition\\
B. Map Initialization\\
C. Monocular SLAM

\subsection{System Overview}
A. Feature Choise: same feature used for tracking and mapping.\\
B. Three threads: Tracking, Local mapping and Loop closing.\\
C. Map points, key frames, and their selections.
Each map point pi stores:
\begin{enumerate}
\item Its 3D position Xwi in the world coordinate system.
\item The viewing direction ni, which is the mean unit vector of all its viewing directions.
\item The maximum dmax and minimum dmin distances at which the points can be observed.
\end{enumerate}
Each keyframe Ki stores:
\begin{enumerate}
\item The camera post Tiw
\item The camera intrinsics, including focal length and principal point
\item All the ORB features extracted in the frame whose coordinates are undistored in a distorted if a distortion model is provided.
\end{enumerate}
D. Covisibility Graph and Essential Graph\\
E. Bags of Words Place Recognition
\subsection{Automatic Map Initialization}
\begin{itemize}
\item Find initial correspondences\\
Extract ORB feature (only at the finest sacle) in the current frame Fc.
\item Parallel computation of the two models
\item Model selection
\item Motion and Structure from Motion recovery
\item Bundle adjustment
\end{itemize}

\subsection{Tracking}
In this sectino we describe the step of tracking thread that are perfromed with every frame form the camera.
\begin{enumerate}
\item ORB Extraction
\item Initial Pose Estimation from Previous Frame
\item Initial Pose Estimatin via Global Relocalization
\item Track Local Map
	\begin{enumerate}
	\item Compute the map point projection x in the current frame. Discard if it lays out of the image bounds.
	\item Compute the angle between the current viewing ray v and the map point mean viewing direction n. Discard if v.n < cos(60 angle).
	\item Compute the distance d from map point to camera center. Discard if it is out of the scale invariance region of the map point d [dmin, dmax].
	\item Compute the scale in the frame by the ratio d/dmin.
	\item Compare the representative descriptor D of the map point with the still unmalted ORB featrues.
	\end{enumerate}
\item New Keyframe Decision
	\begin{enumerate}
	\item More than 20 frames must have passed from the last global relocalization.
	\item Local mapping is idel, or more than 20 frame have passed from last keyframe insertion.
	\item Current frame tracks at lest 50 points.
	\item Current frame tracks less than 90 percent points than Kref.
	\end{enumerate}
\item New Keyframe Decision
\end{enumerate}

\subsection{Local Mappting}

\begin{enumerate}
\item KeyFrame Insertion
\item Recent Map Points Culling
	\begin{enumerate}
	\item The tracking must find the point in more than the 25 percent of the frames in which it is predicted to be visible.
	\item If more than one keyframe has passed from map point creation, it must be observed from at least three keyframes.
	\end{enumerate}
\item New Map Point Creation
\item Local Bundle Adjustment
\item Local Keyframe Culling
\end{enumerate}

\subsection{Loop Closing}

\begin{enumerate}
\item Loop Candidates Detection
\item Compute the Similarity Transformation
\item Essential Graph Optimization
\end{enumerate}




\end{document}
